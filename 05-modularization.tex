\documentclass{article}
\usepackage{booktabs}
\usepackage{graphicx}
\usepackage{amsmath}
\usepackage{color}
\usepackage{hyperref}

\begin{document}

\title{Latex Essential Training}
\author{Ravi Kumar Tiwari}
\date{\today}
\maketitle



\begin{abstract}
This document describes the use of latex for creating a document. The document is divided into various sections that describe
how to add and/or customize various parts of a document. For example, In order to add abstract, we create a new environment 
using \verb+ \begin{abstract} + and \verb+ \end{abstract} + and put the contents of the abstract inside the abstract environment.
\end{abstract}

\newpage
\tableofcontents

\newpage
\listoftables
\listoffigures
\newpage

\section{Adding Section}
To add a section just use \verb+\section +

\subsection{Adding Subsection}
The command to add a subsection is very similar to that of section. The only difference being instead of 
section we write subsection \verb+\subsection +

\subsubsection{Adding Sub Subsection}
The above method can be further extended to add sub subsection. 

\input{"06-addingContentFromaFile.tex"}

\input{"wordFormatting.tex"}

\input{"list.tex"}

\input{"table.tex"}

\input{"figure.tex"}

\input{"tableReference.tex"}

\input{"figReference.tex"}

\input{"maths.tex"}

\input{"crossReferencing.tex"}

\input{"citation.tex"}



 \begin{thebibliography}{1}

  \bibitem{tex1} D. E. Knuth {\em The tex book }  Addison Wesley 1986.

  \bibitem{tex2}  D. E. Knuth {\em Typesetting concrete mathematics, TUGboat}, April 1989

  \end{thebibliography}












\end{document}