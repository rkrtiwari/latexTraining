\section{Word Formatting}

\subsection{Bold}
To turn a word or a group of words into bold, use the command: \verb+\textbf{text}+
For example, the word after this: \textbf{have been turned into bold using the above command}

\subsection{Italics}
To italicise a word or a group of words, use the command: \verb+\textit{text}+
For example, the word after this: \textit{have been italicized using the above command}

\subsection{Underline}
To underline a word or a group of words, use the command: \verb+\underline{text}+
For example, the word after this: \underline{have been italicized using the above command}

\subsection{Color Font}
\textcolor{red}{This is typed in red}

\subsection{Font Size}
The following command can be used to change the font sizes: \verb+{\tiny text}+,
\verb+{\small text}+, \verb+{\large text}+, \verb+{\huge text}+. For example:
{\tiny tiny}, {\small small}, normal, {\large large},  {\huge huge} \\

You can also choose the font size using \verb+\fontsize{<size>}{<line space>}+ command.
For example. {\fontsize{2cm}{2.5cm}{\selectfont This is big}}

\subsection{Changing font side by defining an environment}
\begin{huge}
All the word have been typed inside huge environment. No wonder the words are huge!!!
\end{huge}

\subsection{Centering Text}
\begin{center}
Everything will be centered. \\
As these lines are typed inside center environment
\end{center}

\subsection{Paragraph in a box}
\parbox{2.5cm}{TUG is an acronym. It means \TeX{} users Group.}


\subsection{Font type}
To use a sans-serif font use the command: \verb+\textsf{text}+
For example, the word after this: \textsf{have been written in sans-serif font}

\subsection{Quote}
Niels Bohr said: ``An expert is a person who has made all the mistakes that can be made in a very narrow field''.
Albert Eienstein said:
\begin{quote}
Anyone who has never made a mistake has never tried anything new
\end{quote}
Errors are inevitable. So, let's be brave trying something new.



\subsection{Start a new line}
To start a new line, \newline use the command: \verb+{\newline}+. You can also use \verb+ {\\}+
\\ for starting a new line.

\subsection{Start a new page}
To start a new page, \newpage use the command: \verb+{\newpage}+.